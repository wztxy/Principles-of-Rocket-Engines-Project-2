\documentclass[zihao=-4]{ctexart}
\usepackage[normalem]{ulem}
\useunder{\uline}{\ul}{}
%********************导言区宏包引入********************
\usepackage{xeCJK}
\usepackage{amssymb}
\usepackage{amsmath}
\usepackage{listings} %代码
\usepackage{graphicx}
\usepackage{multicol} %回车换段
\usepackage{xcolor}
\usepackage{geometry} %页面设置
\usepackage{fontspec}
\usepackage{setspace}
\usepackage{times}
\usepackage{fancyhdr} %页眉页脚
\pagestyle{fancy}
\usepackage{siunitx} % 推荐用于单位和数字格式化
\sisetup{detect-all=true, per-mode=symbol, group-separator={\,}, group-minimum-digits=4}
\usepackage{float} %表格位置
\usepackage{titlesec} %设置
\usepackage{titletoc}
\usepackage{ctex}
\usepackage{gbt7714}    %控制参考文献格式为国标
\usepackage{multirow}
\usepackage{booktabs}   %表格相关
\usepackage{setspace}   %设置行距
\usepackage{caption} %caption
\usepackage{subcaption} %子图的caption
\usepackage{changepage} %左右缩进
\usepackage{xcolor}
\usepackage{xurl}                 % 允许 URL 在任意合适位置断行
\usepackage[hidelinks]{hyperref}  % 若已加载 hyperref,请只确保 xurl 在它之前加载
\usepackage{tabularx} % 新增:自适应表格列宽


\newcolumntype{Y}{>{\centering\arraybackslash}X} % 居中可伸缩列

\lstset{
    basicstyle=\ttfamily\small,
    keywordstyle=\color{blue},
    commentstyle=\color{gray},
    stringstyle=\color{black},
    numbers=left,
    numberstyle=\tiny,
    breaklines=true,
    captionpos=b,
    frame = shadowbox,
    rulesepcolor=\color{red!20!green!20!blue!20},
    framexleftmargin=2em
}

\graphicspath{ {include_picture/} }
\let\algorithm\relax
\let\endalgorithm\relax
\usepackage[ruled,vlined]{algorithm2e}%[ruled,vlined]{
\usepackage{algpseudocode}
\renewcommand{\algorithmicrequire}{\textbf{Input:}} 
\renewcommand{\algorithmicensure}{\textbf{Output:}}
%\renewcommand\thepage{\zihao{-5} ~\arabic{page}~}%页码字号

%定义两个arg
\DeclareMathOperator*{\argmax}{arg\,max}
\DeclareMathOperator*{\argmin}{arg\,min}
\DeclareCaptionLabelSeparator{mysep}{\space\space}  %自定义caption格式
\captionsetup[figure]{font={small}, labelfont=bf, labelsep=mysep, textfont=bf}   %图片caption格式
\captionsetup[table]{font={small}, labelfont=bf, labelsep=mysep, textfont={bf}}   %表格caption格式
\bibliographystyle{gbt7714-numerical} %修改了title斜体内容

%********************导言区宏包引入********************
%********************第三方字体引入********************
%\setCJKmainfont[Path=fonts/,BoldFont=simhei.ttf,ItalicFont=simkai.ttf,SlantedFont=simfang.ttf]{simsun.ttc}
%中文字体涵盖黑体、宋体、楷体、仿宋
\setmainfont[Path=fonts/, 
BoldFont = times-new-roman-bold.ttf,
ItalicFont = times-new-roman-italic.ttf,
BoldItalicFont = times-new-roman-bold-italic.ttf
]{times-new-roman.ttf}
\setmonofont[Path=fonts/]{Courier New.ttf}
\setCJKfamilyfont{hwzs}[Path=fonts/]{STKzhongsong.ttf}%使用STZhogsong华文中宋字体
\newcommand{\zhongsong}{\CJKfamily{hwzs}}
\setCJKfamilyfont{hwxw}[Path=fonts/]{STKxinwei.ttf} % XSP 2023/3/3:
\newcommand{\xinwei}{\CJKfamily{hwxw}}              %  使用STZxinwei华文新魏字体.

%********************第三方字体引入********************


%********************中文字号设置********************
%\newcommand{\chuhao}{\fontsize{42pt}{\baselineskip}\selectfont}
\newcommand{\chuhao}{\fontsize{42pt}{0}}
\newcommand{\xiaochu}{\fontsize{36pt}{0}}
\newcommand{\yihao}{\fontsize{28pt}{0}}
\newcommand{\erhao}{\fontsize{21pt}{0}}
\newcommand{\xiaoer}{\fontsize{18pt}{0}}
\newcommand{\sanhao}{\fontsize{16pt}{0}}
\newcommand{\sihao}{\fontsize{14pt}{0}}
\newcommand{\xiaosi}{\fontsize{12pt}{0}}
\newcommand{\wuhao}{\fontsize{10.5pt}{0}}
\newcommand{\xiaowu}{\fontsize{9pt}{0}}
\newcommand{\liuhao}{\fontsize{8pt}{0}}
\newcommand{\qihao}{\fontsize{5.25pt}{0}}
%********************中文字号设置********************


%********************页边距设置********************
\geometry{left=3cm,right=2cm,top=2.5cm,bottom=2.5cm}
\geometry{a4paper} % xsp 2023/3/7: 调整纸张大小为A4
%********************页边距设置********************

%********************段间距设置********************
\newcommand{\setParDis}{\setlength {\parskip} {0pt} }
%请在每部分使用这个
%********************段间距设置********************

%********************

\begin{document}
%********************页眉页脚设置********************
\lhead{}%设置左页眉为空
\rhead{}%设置左页眉为空
\setlength{\headwidth}{\textwidth}% 2023/3/3 XSP: 页眉长度适应文本
%********************页眉页脚设置********************


%********************标题格式设置********************

%\setcounter{secnumdepth}{0}%该命令取消了章标题前数字label

\CTEXsetup[name={,、},number={\chinese{section}}]{section}
\CTEXsetup[name={(,)},number={\chinese{subsection}}]{subsection}
\CTEXsetup[name={,.},number={\arabic{subsubsection}}]{subsubsection}% 不加会导致目录格式错误
% 设置subsubsection等格式
% \titleformat{\section}[block]{\sanhao\bfseries\centering}{\chinese{section}、}{0pt}{}[]
% \titleformat{\subsection}[block]{\sihao\bfseries}{(\chinese{subsection})}{0pt}{}[]
% \titleformat{\subsubsection}[block]{\xiaosi\bfseries}{\arabic{subsubsection}、}{0pt}{}[]
\titleformat{\section}[block]{\sanhao\heiti\centering}{\chinese{section}、}{0pt}{}[]    % XSP 2023/3/3:
\titleformat{\subsection}[block]{\sihao\heiti}{(\chinese{subsection})}{0pt}{}[]       %   将正文标题字体由加粗
\titleformat{\subsubsection}[block]{\xiaosi\heiti}{\arabic{subsubsection}.}{0pt}{}[]   % 修改为黑体。
\titlespacing{\section}{0pt}{25pt}{12pt}
\titlespacing{\subsection}{0pt}{7pt}{7pt}
\titlespacing{\subsubsection}{0pt}{5pt}{4pt}

\titlecontents{section}[1.6em]{\addvspace{2pt}\filright}
{\contentspush{\thecontentslabel\hspace{0.8em}}}
{}{\titlerule*[8pt]{.}\contentspage}

\titlecontents{subsection}[3.2em]{\addvspace{2pt}\filright}
{\contentspush{\thecontentslabel\hspace{0.8em}}}
{}{\titlerule*[8pt]{.}\contentspage}

\titlecontents{subsubsection}[6.4em]{\addvspace{2pt}\filright}
{\contentspush{\thecontentslabel\hspace{0.8em}}}
{}{\titlerule*[8pt]{.}\contentspage}
%********************标题格式设置********************

%\setcounter{section}{-3}  %标题计数器
%\stepcounter{section}

%*******************行间距段前段后*******************
\linespread{1.8}
%行间距为实际行间距乘以1.2,如此处实际为1.5倍行距
\setlength{\parskip}{0.5\baselineskip}
%*******************行间距段前段后*******************



%********************封面部分********************
%
%     论文题目:应准确、鲜明、简洁,能概括整个论文中最主要和最重要的内容。
% 题目不超过20个中文字,若语意未尽,可用副标题补充说明。副标题应处于从属
% 地位,一般可在题目的下一行用破折号“——”引出。论文题目应避免使用不常用缩
% 略词、首字母缩写字、字符、代号和公式等。
%
\def\Fengru{第三十五届“冯如杯”竞赛主赛道}
\leftline{\includegraphics[scale=1]{include_picture/xiaohui.png}} % XSP 2023/3/3: 取消校徽段首缩进
%格式控制部分
% \par \  
% \par \
% \par \
\vspace{32pt}
\begin{center}
\includegraphics[height=2.25cm, width=12.78cm, scale=1]{include_picture/xiaoming.png}
\end{center}
%格式控制部分
\vspace{12pt}

\begin{spacing}{3}
    % \erhao
    \begin{center}
      {
        \fontsize{22pt}{3}\selectfont
        \zhongsong{SLS系列运载火箭性能热力计算研究报告} %黑体这样调用,其余字体同理
      } 
        % \zhongsong{“冯如杯”竞赛主赛道项目是什么}
    \end{center}
    %\rightline{\xinwei\sanhao{——基于 Latex 的论文模板}} % XSP 2023/3/3: 副标题二号华文新魏居右
\end{spacing}
%格式控制部分
% \par \ 
% \par \
\par \ 
\par \
\par \ 
\par \
% \begin{center}
%     \sihao
%     \textbf{学院:计算机学院}
%     \par \ 
%     \textbf{本模板原作者:Someday}
% \end{center}

%格式控制部分
\par \ 
\begin{center}
\sanhao
\par \
\par \
\par \
\par \
\par \
\par \
\centerline{\heiti{2025年11月}}%封面年月去掉
\end{center}

\pagenumbering{gobble} %封面无页码
%\thispagestyle{empty}


\renewcommand{\headrulewidth}{0pt}%没有页眉装饰线
\clearpage
\pagenumbering{roman} %摘要目录页小写罗马

\xiaosi
%********************摘要部分********************
\section*{摘要}
\begin{spacing}{1.5}
  \setParDis %设置段间距为 0
  太空发射系统(Space Launch System,SLS)是美国国家航空航天局(NASA)为阿尔忒弥斯(Artemis)载人登月与深空探测任务研制的超重型运载火箭系列。作为当今唯一具备在单次发射中将“猎户座”(Orion)飞船及货物直接送往月球轨道能力的火箭,SLS 的推进系统设计在技术与性能层面代表了当前化学火箭的最高水平(NASA,2024)。本研究基于 NASA 官方资料与技术报告,对 SLS 系列火箭的推力结构与发动机性能进行定量分析。

SLS 核心级由四台 RS-25 液氢液氧发动机组成,每台海平面推力约 1860 kN,真空比冲达 452 s,总推力超过 7,400 kN(NASA Core Stage Data, 2023)。助推系统采用两台五节固体火箭助推器(SRB),每台提供约 16,000 kN 推力,在起飞阶段贡献总推力的约 75\%(NASA Booster Reference, 2023)。随着 Block 1B 与 Block 2 构型的演进,计划采用更高性能的探索上面级(EUS)与改进型 RS-25E 发动机,以实现超过 130 吨级的低地轨道运载能力与更高的月球转移推力。

研究结果表明,SLS 系列在推力集成与发动机复用技术上兼具高性能与工程成熟度,其发动机系统的高比冲与可控推力矢量能力为深空载人任务提供了可靠动力支持。然而,其高成本与有限复用性仍制约其商业可持续性。
\end{spacing}
    
\textbf{关键词:}SLS 运载火箭, RS-25 发动机, 固体助推器,推力性能, NASA

\newpage
\section*{\textbf{Abstract}} % XSP 2023/3/8: Abstract 加粗
\begin{spacing}{1.5}
\begin{adjustwidth}{0.42cm}{0.42cm}
  \setParDis %设置段间距为 0

\qquad The Space Launch System (SLS), developed by the National Aeronautics and Space Administration (NASA) for the Artemis lunar and deep-space exploration missions, represents the most advanced heavy-lift launch vehicle in operation today. Designed as the only rocket capable of sending the Orion spacecraft and its payload directly to lunar orbit in a single launch, SLS exemplifies the current peak of chemical propulsion technology (NASA, 2024). This study focuses on a quantitative analysis of the SLS propulsion architecture and engine performance, based primarily on official NASA documentation and technical data.

The SLS Core Stage employs four RS-25 liquid hydrogen and liquid oxygen engines, each producing approximately 1,860 kN of sea-level thrust with a vacuum specific impulse of 452 s, yielding a combined thrust exceeding 7,400 kN (NASA Core Stage Data, 2023). Two five-segment solid rocket boosters (SRBs) provide an additional 16,000 kN each, contributing roughly 75\% of the total thrust during liftoff (NASA Booster Reference, 2023). With the evolution from Block 1 to Block 1B and Block 2 configurations, the system integrates the Exploration Upper Stage (EUS) and upgraded RS-25E engines, enabling low Earth orbit (LEO) payload capacities exceeding 130 metric tons and higher translunar injection (TLI) thrust performance.

Results indicate that the SLS series achieves high performance and engineering maturity in thrust integration and engine vector control, providing reliable propulsion for crewed deep-space missions. However, its high cost and limited reusability remain major constraints on its long-term commercial viability.

\textbf{Keywords: Space Launch System (SLS), RS-25 engine, solid rocket booster, thrust performance, NASA}

\end{adjustwidth}
\end{spacing}





%********************目录部分********************
\clearpage
\tableofcontents
\clearpage




\renewcommand{\headrulewidth}{0.4pt} %恢复页眉装饰线

%********************正文页眉部分********************
%\lhead{} 
\chead{\xiaowu 北京航空航天大学{\small 2025-2026}秋季《火箭发动机原理》课程大作业} %设置居中页眉
%********************正文页眉部分********************

\pagenumbering{arabic} %正文页码从1开始,用阿拉伯数字
\setcounter{page}{1} 

\section{简介}
\setParDis %设置段间距为 0
\begin{spacing}{1.5}
尽管SLS的宏观运载能力与任务规划已在总体设计层面明确,但其核心竞争力的根源在于推进系统的能量转换效率。SLS核心级搭载的RS-25液氢液氧发动机与上面级(EUS/ICPS)选用的RL10系列发动机,其内部复杂的热力过程直接决定了火箭的有效载荷与比冲性能。

然而,在实际工程应用与全飞行包络分析中,火箭发动机的性能受多种耦合因素影响,包括但不限于喷管的非等熵流动损失、两相流效应、燃烧室内的非化学平衡流动以及飞行高度变化带来的背压波动。由于涉及的边界条件极为复杂且部分工程经验参数难以通过公开资料精确获取,直接进行包含所有损失项的工程数值计算往往存在较大的不确定性,难以建立统一的评价基准。

鉴于此,本报告的热力计算部分将剥离复杂的工程扰动因素,建立在理想火箭发动机的热力学模型基础之上。本次计算的核心策略是:将燃烧室压力($p_c$)与氧化剂/燃料混合比($r_{of}$)设定为关键控制变量。
通过锁定这两个决定燃烧室热力状态的核心参数,利用化学平衡计算方法,本报告旨在精确求解在冻结流或平衡流假设下的理论燃烧温度、特征速度、燃气化学组成、分子量以及不同膨胀比下的真空比冲等理想性能参数。这种基于确定性输入的计算方法,不仅能够准确复现SLS动力系统在设计工况下的理论性能极限,也为后续评估实际飞行中的效率损失系数提供了坚实的基准参考。
\end{spacing}

\begin{figure}[H] %H为当前位置,!htb为忽略美学标准,htbp为浮动图形
    \centering %图片居中
    \includegraphics[width=0.8\textwidth]{SLS“四年计划”.png} %插入图片,[]中设置图片大小,{}中是图片文件名
    \caption{SLS“四年计划”} %最终文档中希望显示的图片标题
    \label{example_label} %用于文内引用的标签
\end{figure}

SLS系列采用分阶段发展的技术路线:Block 1主要用于初始验证任务,配备过渡型低温推进级(ICPS);Block 1B换装四台RL10C-3发动机的探索上面级(EUS),显著提升了深空运载性能;Block 2则计划采用更先进的助推器与改进型RS-25E发动机,实现最大化推力与结构效率。SLS的设计融合了航天飞机成熟的RS-25液体主发动机与五段式固体助推器技术,在推进系统集成、推力调节范围、燃烧效率及热防护材料等方面均进行了系统性创新。

从国际视角来看,SLS的成功不仅重新确立了美国在深空载人航天领域的主导地位,也推动了全球重型运载火箭技术的再平衡格局。对于中国而言,研究SLS的发展路径与性能特征具有重要的战略与工程意义。首先,SLS在液体发动机再利用、深冷推进剂储输与大推力结构稳定性控制方面的经验,为我国新一代重型运载火箭(如长征九号)的研制提供了重要参考。其次,SLS在任务体系设计上通过多构型演进实现持续适应性,其“模块化推进级”和“任务可配置上面级”思路,对我国未来载人登月与深空探测发射架构优化具有启发价值。最后,其在国际合作、商业航天接口设计及任务管理体制方面的探索,也为我国构建高效、可持续的航天发射体系提供了可借鉴的模式。

本报告以SLS系列运载火箭为研究对象,基于NASA官方技术文档、任务规划报告及国际学术文献,系统分析其推力系统结构、发动机性能参数、构型演化规律及其技术创新特点。研究旨在从工程与战略两个维度探讨SLS的发展逻辑及其对深空探测体系的支撑作用,同时评估其在推进效率、系统可靠性与经济可持续性方面的优劣势。通过对SLS系列的深入剖析,本文期望为我国重型运载火箭总体设计、发动机技术发展及未来深空任务的体系构建提供有价值的参考与启示。

\section{研究背景与目的}
\subsection{研究背景}
随着“阿尔忒弥斯”计划的推进,人类航天探索的重心正从近地轨道(LEO)全面转向月球及更深远的深空。作为这一战略转型的核心载体,SLS不仅继承了航天飞机时代的成熟技术遗产(如RS-25发动机和固推技术),更通过深度改进与系统集成,成为了当今世界推力最强、技术最复杂的现役运载火箭。

然而,重型运载火箭的研制不仅仅是结构尺寸的放大,其核心在于动力系统能量转换效率的极致追求。SLS所采用的液氢液氧(LH2/LOX)推进体系,代表了化学推进技术的高能巅峰。RS-25发动机的分级燃烧循环与RL10发动机的膨胀循环设计,均是在极端的燃烧室压力与混合比条件下工作的。虽然NASA公布了SLS的宏观运载指标,但关于其动力系统内部精细的热力学匹配机制、能量释放效率以及在不同工况下的理论性能极限,往往缺乏详尽的公开解析。

在缺乏内部工程数据的情况下,通过热力学基本原理,利用燃烧室压力与混合比这两个关键控制参数进行反向建模与正向计算,是深入理解SLS动力系统“心脏”机理的必经之路。这种基于第一性原理的分析,能够帮助我们透过宏观的发射现象,洞察其背后的物理本质。

\subsection{研究目的}
本报告旨在通过对SLS动力系统进行基于化学平衡流/冻结流理论的热力计算,实现以下多重目的:
\subsubsection{理论验证与参数复现(基础目的)}
针对SLS关键的RS-25及RL10发动机,在工程参数存在模糊性的客观条件下,本研究采取“锁定核心变量”的策略,即通过确定的燃烧室压力($P_c$)和混合比(MR),精确计算其真空比冲、特征速度、燃烧温度及燃气组分等理想热力参数。这一过程旨在建立一个标准的\textbf{“理论性能基线”},验证热力学模型在分析顶尖液体火箭发动机时的适用性,并为后续分析非理想损失提供量化参考。

\subsubsection{学术训练与理论实践(学习意义)}
火箭发动机热力计算是连接《工程热力学》、《气体动力学》与《火箭推进原理》的桥梁。通过本次计算研究,能够将书本上抽象的焓熵图解、化学平衡方程以及拉伐尔喷管流动理论,转化为具体的工程分析能力。

这不仅加深了对\textbf{“高压补燃循环”与“氢氧高比冲特性”}的理解,更重要的是培养了在有限数据条件下进行工程估算与系统分析的思维模式。掌握这种从“输入条件”到“性能输出”的逻辑链条,是未来从事航天工程设计与研究的重要基石。

\subsubsection{对我国重型运载火箭的借鉴与启示(现实意义)}
当前,我国正处于航天强国建设的关键时期,新一代载人运载火箭(长征十号)与重型运载火箭(长征九号)的研制工作正在紧锣密鼓地进行。SLS作为目前唯一已成功飞行的登月级重型火箭,其动力系统的设计哲学对我国具有极高的参考价值:

\begin{itemize}
    \item \textbf{技术路线的对标:}通过计算分析RS-25的高压高性能特性,可以为我国正在攻关的大推力氢氧发动机提供性能对标数据,分析中美在氢氧发动机比冲与推重比上的技术差异与追赶方向。
    \item \textbf{系统匹配的思考:}SLS采用“固液混合起飞级 + 高效氢氧芯级 + 氢氧上面级”的动力搭配。通过热力计算分析其各级发动机的性能跨度,有助于理解其如何平衡起飞推力与入轨比冲的关系,为我国重型火箭在级间比选、混合比优化以及弹道设计方面提供理论支撑与决策参考。
\end{itemize}

综上所述,本研究不仅是对SLS这一具体型号的性能复盘,更是对高性能液体火箭发动机热力学本质的一次深度探索,兼具学术训练价值与工程借鉴意义。




\section{研究方法}
\subsection{计算原理}
\subsubsection{计算方法}
\subsubsection{参数确定}
\subsection{程序设计原理}
\subsubsection{设计思路}
\subsubsection{关键算法}
\subsection{性能评价方法}
\subsubsection{横向对比验证}
\subsubsection{自身对比验证}
\section{研究结果}
\subsection{程序使用说明}
\subsection{性能计算结果}

\section{结论}% section*生成无标号章节
\subsection{SLS性能评估}
\subsection{程序功能性能总结}




\section*{组员分工说明}
\addcontentsline{toc}{section}{组员分工说明}

\textbf{组长}:周河

\textbf{组员}:付栋博,苏震,张旭玚,邓淼,于正泽

\begin{itemize}
  \item 付栋博,邓淼:文献调研与资料整理
  \item 张旭玚,于正泽:算法与程序开发、测试
  \item 周河:PPT制作与统筹推进
  \item 苏震:报告撰写与排版
\end{itemize}
团队成员相互讨论、出谋划策、共同解决问题,致谢每一位组员与助教老师的辛勤付出。
\newpage

% XSP 2023/3/16: bib支持不全,暂时改为手动
\section*{参考文献} % section*生成无标号章节题目
\addcontentsline{toc}{section}{参考文献} % 将无标号章节添加至目录
% 著作: [序号]作者.书名[标识码].出版地:出版社,出版年.
%[1]张绿云,杨开,王林.美国“阿尔忒弥斯”计划运载能力分析[J].国际太空,2025,(05):56-62.

% 译著: [序号]国名或地区(用圆括号)原作者.书名[标识码].译者.出版地:出版社,出版年.
%[2](英)霭理士.性心理学[M].潘光旦译.北京:商务印书馆,1997.

% 古典文献 文史古籍类引文后加序号,再加圆括号,内加注书名、篇名

% 论文集: [序号]编者.书名[标识码].出版地:出版社,出版年.
%[3]伍蠡甫.西方论文选(下册)[C].上海:上海译文出版社,1979.

% 期刊文章: [序号]作者.篇名[标识码].刊名,年,(期).
[1]张绿云,杨开,王林.美国“阿尔忒弥斯”计划运载能力分析[J].国际太空,\url{2025,(05):56-62.}

[2]王晓明,毛利民,孟昭龙.浅析SLS移动发射台及其研制情况[J].导弹与航天运载技术(中英文),2024,(04):70-76.

[3]张绿云,杨开,王林,等.美国SLS-1重型运载火箭首飞简析[J].国际太空,\url{2023,(01):48-}

[4]张绿云,才满瑞,杨开,等.美国“航天发射系统”重型火箭成功首飞及其发展分析[J].中国航天,2022,(11):35-41.

[5]张绿云,杨开.美国SLS重型运载火箭研制特点分析[J].国际太空,2021,(11):43-47.

[6]张绿云,龙雪丹,黄长梅,等.国外新一代重型运载火箭发展分析[J].国际太空,\url{2021,(05):26-31.}

[7]张绿云,曲晶.美国SLS重型火箭首次芯级点火试验简析[J].国际太空,2021,(02):8-13.

[8]张绿云.2019年美国SLS重型运载火箭研制进展[J].国际太空,2020,(03):30-36.

[9]任奇野,曲晶.美国SLS重型运载火箭最新进展分析[J].国际太空,2018,(05):58-65.

[10]中国航天工业总公司,《世界导弹与航天发动机大全》编辑委员会编.世界导弹与航天发动机大全.军事科学出版社.1999

[11]梁国柱,火箭发动机原理[M].北京:北京航空航天大学出版社,2025.

[12]NASA. SLS (Space Launch System) Solid Rocket Booster[EB/OL].(2024-7-25) [2025-9-24].\url{https://www.nasa.gov/reference/sls-space-launch-system-solid-rocket-booster/}

[13]NASA. SLS (Space Launch System) RS-25 Core Stage Engine[EB/OL].(2025-3) [2025-9-24].\url{https://www.nasa.gov/wp-content/uploads/2025/04/sls-4963-sls-rs-25-engine-fact-sheet-508.pdf?emrc=bb1960
}

[14]NASA. Space Launch System RL10 Engine [EB/OL].(2022-1) [2025-9-24]. 
\url{https://www.nasa.gov/wp-content/uploads/2023/09/rl10-fact-sheet-02082022-final.pdf?emrc=b590f7}

[15]NASA. SLS (Space Launch System) Launch Vehicle Stage Adapter [EB/OL].(2024-8) [2025-9-24].\url{https://www.nasa.gov/reference/space-launch-system-launch-vehicle-stage-adapter-lvsa/}

[16]NASA. SLS (Space Launch System) Interim Cryogenic Propulsion Stage [EB/OL].\\(2024-11-29) [2025-9-24]. https://www.nasa.gov/reference/icps/

[17]NASA. SLS (Space Launch System) Block 1B[EB/OL].\url{(2024-7-3) [2025-9-24]. 
https://www.nasa.gov/reference/sls-space-launch-system-block-1b/}

[18]NASA. Space Launch System Exploration Upper Stage (EUS) [EB/OL].(2023-2) [2025-9-24].\url{https://www.nasa.gov/reference/space-launch-system-exploration-upper-stage-eus/}

[19]Naveen Vetcha, Matthew B. Strickland, Kenneth D. Philippart and Thomas V. Giel. "Overview of RS-25 Adaptation Hot-Fire Test Series for SLS, Status and Lessons Learned," AIAA 2018-4459. 2018 Joint Propulsion Conference. July 2018.

% 报纸文章: [序号]作者.篇名[标识码].报纸名,出版日期(版次)
%[5]谢希德.创造学习的新思路[N].人民日报,1998-12-25(10)

% 外文文献: 要求外文文献所表达的信息和中文文献一样多,但文献类型标识码可以不标出。
%[6]Mansfeld, R.S. \& Busse. \textit{T.V. The Psychology of creativity and discovery}, Chinago:
%NelsonHall, 1981




% \begingroup
% \setstretch{2.0}    %行距2
% \setlength{\bibsep}{0pt}    %段前段后0
% \begin{adjustwidth}{0.42cm}{0.42cm} %左右缩进0.42cm
% \bibliography{references}
% \end{adjustwidth}
% \endgroup

\end{document}
