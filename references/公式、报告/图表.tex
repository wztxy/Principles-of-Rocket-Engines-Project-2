\documentclass[zihao=-4]{ctexart}
\usepackage[normalem]{ulem}
\useunder{\uline}{\ul}{}
%********************导言区宏包引入********************
\usepackage{xeCJK}
\usepackage{amssymb}
\usepackage{amsmath}
\usepackage{listings} %代码
\usepackage{graphicx}
\usepackage{multicol} %回车换段
\usepackage{xcolor}
\usepackage{geometry} %页面设置
\usepackage{fontspec}
\usepackage{setspace}
\usepackage{times}
\usepackage{fancyhdr} %页眉页脚
\pagestyle{fancy}
\usepackage{siunitx} % 推荐用于单位和数字格式化
\sisetup{detect-all=true, per-mode=symbol, group-separator={\,}, group-minimum-digits=4}
\usepackage{float} %表格位置
\usepackage{titlesec} %设置
\usepackage{titletoc}
\usepackage{ctex}
\usepackage{gbt7714}    %控制参考文献格式为国标
\usepackage{multirow}
\usepackage{booktabs}   %表格相关
\usepackage{setspace}   %设置行距
\usepackage{caption} %caption
\usepackage{subcaption} %子图的caption
\usepackage{changepage} %左右缩进
\usepackage{xcolor}
\usepackage{xurl}                 % 允许 URL 在任意合适位置断行
\usepackage[hidelinks]{hyperref}  % 若已加载 hyperref,请只确保 xurl 在它之前加载
\usepackage{tabularx} % 新增:自适应表格列宽


\newcolumntype{Y}{>{\centering\arraybackslash}X} % 居中可伸缩列

\lstset{
    basicstyle=\ttfamily\small,
    keywordstyle=\color{blue},
    commentstyle=\color{gray},
    stringstyle=\color{black},
    numbers=left,
    numberstyle=\tiny,
    breaklines=true,
    captionpos=b,
    frame = shadowbox,
    rulesepcolor=\color{red!20!green!20!blue!20},
    framexleftmargin=2em
}

\graphicspath{ {include_picture/} }
\let\algorithm\relax
\let\endalgorithm\relax
\usepackage[ruled,vlined]{algorithm2e}%[ruled,vlined]{
\usepackage{algpseudocode}
\renewcommand{\algorithmicrequire}{\textbf{Input:}} 
\renewcommand{\algorithmicensure}{\textbf{Output:}}
%\renewcommand\thepage{\zihao{-5} ~\arabic{page}~}%页码字号

%定义两个arg
\DeclareMathOperator*{\argmax}{arg\,max}
\DeclareMathOperator*{\argmin}{arg\,min}
\DeclareCaptionLabelSeparator{mysep}{\space\space}  %自定义caption格式
\captionsetup[figure]{font={small}, labelfont=bf, labelsep=mysep, textfont=bf}   %图片caption格式
\captionsetup[table]{font={small}, labelfont=bf, labelsep=mysep, textfont={bf}}   %表格caption格式
\bibliographystyle{gbt7714-numerical} %修改了title斜体内容

%********************导言区宏包引入********************
%********************第三方字体引入********************
%\setCJKmainfont[Path=fonts/,BoldFont=simhei.ttf,ItalicFont=simkai.ttf,SlantedFont=simfang.ttf]{simsun.ttc}
%中文字体涵盖黑体、宋体、楷体、仿宋
\setmainfont[Path=fonts/, 
BoldFont = times-new-roman-bold.ttf,
ItalicFont = times-new-roman-italic.ttf,
BoldItalicFont = times-new-roman-bold-italic.ttf
]{times-new-roman.ttf}
\setmonofont[Path=fonts/]{Courier New.ttf}
\setCJKfamilyfont{hwzs}[Path=fonts/]{STKzhongsong.ttf}%使用STZhogsong华文中宋字体
\newcommand{\zhongsong}{\CJKfamily{hwzs}}
\setCJKfamilyfont{hwxw}[Path=fonts/]{STKxinwei.ttf} % XSP 2023/3/3:
\newcommand{\xinwei}{\CJKfamily{hwxw}}              %  使用STZxinwei华文新魏字体.

%********************第三方字体引入********************


%********************中文字号设置********************
%\newcommand{\chuhao}{\fontsize{42pt}{\baselineskip}\selectfont}
\newcommand{\chuhao}{\fontsize{42pt}{0}}
\newcommand{\xiaochu}{\fontsize{36pt}{0}}
\newcommand{\yihao}{\fontsize{28pt}{0}}
\newcommand{\erhao}{\fontsize{21pt}{0}}
\newcommand{\xiaoer}{\fontsize{18pt}{0}}
\newcommand{\sanhao}{\fontsize{16pt}{0}}
\newcommand{\sihao}{\fontsize{14pt}{0}}
\newcommand{\xiaosi}{\fontsize{12pt}{0}}
\newcommand{\wuhao}{\fontsize{10.5pt}{0}}
\newcommand{\xiaowu}{\fontsize{9pt}{0}}
\newcommand{\liuhao}{\fontsize{8pt}{0}}
\newcommand{\qihao}{\fontsize{5.25pt}{0}}
%********************中文字号设置********************


%********************页边距设置********************
\geometry{left=3cm,right=2cm,top=2.5cm,bottom=2.5cm}
\geometry{a4paper} % xsp 2023/3/7: 调整纸张大小为A4
%********************页边距设置********************

%********************段间距设置********************
\newcommand{\setParDis}{\setlength {\parskip} {0pt} }
%请在每部分使用这个
%********************段间距设置********************

%********************

\begin{document}
%********************页眉页脚设置********************
\lhead{}%设置左页眉为空
\rhead{}%设置左页眉为空
\setlength{\headwidth}{\textwidth}% 2023/3/3 XSP: 页眉长度适应文本
%********************页眉页脚设置********************


%********************标题格式设置********************

%\setcounter{secnumdepth}{0}%该命令取消了章标题前数字label

\CTEXsetup[name={,、},number={\chinese{section}}]{section}
\CTEXsetup[name={(,)},number={\chinese{subsection}}]{subsection}
\CTEXsetup[name={,.},number={\arabic{subsubsection}}]{subsubsection}% 不加会导致目录格式错误
% 设置subsubsection等格式
% \titleformat{\section}[block]{\sanhao\bfseries\centering}{\chinese{section}、}{0pt}{}[]
% \titleformat{\subsection}[block]{\sihao\bfseries}{(\chinese{subsection})}{0pt}{}[]
% \titleformat{\subsubsection}[block]{\xiaosi\bfseries}{\arabic{subsubsection}、}{0pt}{}[]
\titleformat{\section}[block]{\sanhao\heiti\centering}{\chinese{section}、}{0pt}{}[]    % XSP 2023/3/3:
\titleformat{\subsection}[block]{\sihao\heiti}{(\chinese{subsection})}{0pt}{}[]       %   将正文标题字体由加粗
\titleformat{\subsubsection}[block]{\xiaosi\heiti}{\arabic{subsubsection}.}{0pt}{}[]   % 修改为黑体。
\titlespacing{\section}{0pt}{25pt}{12pt}
\titlespacing{\subsection}{0pt}{7pt}{7pt}
\titlespacing{\subsubsection}{0pt}{5pt}{4pt}

\titlecontents{section}[1.6em]{\addvspace{2pt}\filright}
{\contentspush{\thecontentslabel\hspace{0.8em}}}
{}{\titlerule*[8pt]{.}\contentspage}

\titlecontents{subsection}[3.2em]{\addvspace{2pt}\filright}
{\contentspush{\thecontentslabel\hspace{0.8em}}}
{}{\titlerule*[8pt]{.}\contentspage}

\titlecontents{subsubsection}[6.4em]{\addvspace{2pt}\filright}
{\contentspush{\thecontentslabel\hspace{0.8em}}}
{}{\titlerule*[8pt]{.}\contentspage}
%********************标题格式设置********************

%\setcounter{section}{-3}  %标题计数器
%\stepcounter{section}

%*******************行间距段前段后*******************
\linespread{1.8}
%行间距为实际行间距乘以1.2,如此处实际为1.5倍行距
\setlength{\parskip}{0.5\baselineskip}


%表1
\begin{table}[H]
\small
\caption{RS-25 与 SRB 计算参数}
\centering
\begin{tabular}{lcc}
\hline
\textbf{参数} & \textbf{RS-25} & \textbf{SRB} \\
\hline
\multicolumn{3}{c}{\textbf{海平面性能参数}} \\
推力 (kN)            & 1635.51   & 13616.38  \\
比冲 (m/s)           & 2816.88   & 2270.04   \\
喷气速度 (m/s)       & 3432.58   & 2259.82   \\
质量流量 (kg/s)      & 580.61    & 5998.29   \\
- 氧化剂流量 (kg/s)  & 497.67    & ——      \\
- 燃料流量 (kg/s)    & 82.94     & ——      \\
特征速度 $c^*$ (m/s) & 1906.61   & 1454.32   \\
推力系数 CF          & 1.477     & 1.561     \\
\multicolumn{3}{c}{\textbf{真空性能参数}} \\
推力 (kN)            & 2057.93   & 14716.97  \\
比冲 (m/s)           & 3544.33   & 2453.53   \\
喷气速度 (m/s)       & 3432.58   & 2259.82   \\
推力系数 CF          & 1.859     & 1.687     \\
\multicolumn{3}{c}{\textbf{效率分析}} \\
燃烧室效率 (\%)      & 99.50     & 96.00     \\
喷管效率 (\%)        & 98.00     & 95.00     \\
总体效率 (\%)        & 97.51     & 91.20     \\
\hline
\end{tabular}
\end{table}

%表2
\begin{table}[H]
\small
\caption{RS-25 与国际/中国发动机推力指标对比}
\centering
\resizebox{\textwidth}{!}{
\begin{tabular}{ccccccc}
\hline  
\textbf{发动机型号} & \textbf{推进剂} & \textbf{推力系数(典型值)} & \textbf{海平面推力} & \textbf{真空推力} & \textbf{海平面比冲} & \textbf{典型循环方式}  \\ 
\hline  
RS-25 & $LOX + LH_2$ & $1.477$ & $1636 kN$ & $2058 kN$  & $287s$ &预燃循环 \\
YF-100(中国) & $LOX + RP-1$ & $1.50 \sim 1.60$ & $1200 kN$  & $1200 kN$  & $300s$ &富氧补燃 \\
YF-77(中国) & $LOX + LH_2$ & $1.60 \sim 1.70$ & $700 kN$  & $750 kN$  & $430s$ &膨胀循环 \\
Raptor(SpaceX) & $LOX + CH_4$ & $1.60 \sim 1.70$ & $1850 kN$  & $2300 kN$  & $330s$ &全流预燃 \\
RD-180(俄罗斯) & $LOX + RP-1$ & $1.55 \sim 1.62$ & $3900 kN$  & $4150 kN$  & $311s$ &双室预燃\\
%发动机型号	推进剂	海平面推力	真空推力	海平面比冲	典型循环方式
%RS-25	LOX + LH₂	~875 kN	~1060 kN	~324 s	预燃循环
%YF-100(中国)	LOX + RP-1	~1200 kN	~1330 kN	~300 s	富氧补燃
%YF-77(中国)	LOX + LH₂	~700 kN	~750 kN	~430 s	膨胀循环
%Raptor(SpaceX)	LOX + CH₄	~1850 kN	~2300 kN	~330 s	全流预燃
%RD-180(俄罗斯)	LOX + RP-1	~3900 kN	~4150 kN	~311 s	双室预燃
\hline
\end{tabular} 
}
\\[2mm]
\textit{数据来源:表中所引用的国内外发动机性能指标,均基于 NASA、CASC 及 SpaceX 等权威机构公开参数,以及《Rocket Propulsion Elements》《航天推进技术》等教材与 AIAA/IAC 技术文献中给出的典型广泛应用值,经归纳比对后提取,用于横向性能分析。}
\end{table}


%单图
\begin{figure}[H] %H为当前位置,!htb为忽略美学标准,htbp为浮动图形
    \centering %图片居中
    \includegraphics[width=0.8\textwidth]{流程图白.png} %插入图片,[]中设置图片大小,{}中是图片文件名
    \caption{程序设计流程图} %最终文档中希望显示的图片标题
    \label{example_label} %用于文内引用的标签
\end{figure}

%多图
\begin{figure}[htbp]
  \begin{subfigure}{0.31\textwidth}
    \includegraphics[width=\linewidth]{SLS1型主要参数.png}
    \caption{SLS1型主要参数} \label{fig:9aaa}
  \end{subfigure}%
  \hspace*{\fill}   % maximize separation between subfigures
  \begin{subfigure}{0.31\textwidth}
    \includegraphics[width=\linewidth]{SLS1B 型总体结构.png}
    \caption{SLS1B 型总体结构} \label{fig:9bbb}
  \end{subfigure}
  \hspace*{\fill}   % maximizeseparation between subfigures
  \begin{subfigure}{0.31\textwidth}
    \includegraphics[width=\linewidth]{SLS芯级结构分解图.png}
    \caption{SLS芯级结构分解图} \label{fig:9ccc}
    \end{subfigure}
\caption{SLS1及SLS1B 型号示意图}
\label{qmix-train}
\end{figure}




%代码
\begin{lstlisting}[language=C,caption={比冲计算程序},label={lst:headers}]
\\主函数选择模块
  int main(void) {
    int choice;
    set_utf8_console();

    print_header();

    while (1) {
        print_main_menu();
        choice = get_user_choice(0, 4);

        switch (choice) {
            case 1:
                analyze_preset_engine();
                break;
                ...
}      
\\预设计算模块中选择计算路径
void analyze_preset_engine(void) {
    int engine_choice;
    EngineData engine;
    PerformanceResults results;

    system("cls||clear");
    printf("\n=== 预设发动机性能分析 ===\n\n");
    printf("1. RS-25 主发动机(液氢液氧)\n");
    printf("2. SRB 固体助推器\n");
    printf("0. 返回主菜单\n\n");
    printf("请选择发动机类型: ");

    engine_choice = get_user_choice(0, 2);

    if (engine_choice == 0)
        return;

    if (engine_choice == 1) {
        engine = get_rs25_data();
        printf("\n已加载 RS-25 发动机数据\n");
    } else {
        engine = get_srb_data();
        printf("\n已加载 SRB 固体助推器数据\n");
    }

    print_separator();
    printf("发动机基本参数:\n");
    print_separator();
    printf("发动机名称:%s\n", engine.name);
    printf("发动机类型:%s\n", engine.type);
    printf("推进剂:%s\n", engine.propellant);
    printf("燃烧室压力:%.2f MPa\n", engine.chamber_pressure / 1e6);
    printf("喷管面积比:%.2f\n", engine.area_ratio);
    printf("比热比 γ:%.3f\n", engine.gamma);
    printf("气体常数 R:%.2f J/(kg·K)\n", engine.gas_constant);
    printf("燃烧室温度:%.2f K\n", engine.chamber_temperature);
    printf("喉部面积:%.4f m²\n", engine.throat_area);
    printf("混合比 O/F:%.2f\n", engine.mixture_ratio);
  ...
  }

\end{lstlisting}




\end{document}